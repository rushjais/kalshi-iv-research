\documentclass{article}
\usepackage[utf8]{inputenc}
\usepackage{amsmath}
\usepackage{amsfonts}
\usepackage{amssymb}
\usepackage{graphicx}

\title{Probabilistic Discrepancies between Kalshi and Traditional Options}
\author{William Walz \& Rushil Jaiswal}
\date{January 2026}

\begin{document}

\maketitle

\section{Introduction}
This project looks at the lead-lag relationship between event based prediction markets (Kalshi) and standard financial derivatives. Treating Kalshi contracts as binary options makes it possible to extract an implied probability signal to compare it to the risk-neutral density from the volatility surface of the S\&P 500.

\section{Mathematical Methodology}

\subsection{Probability Extraction from Prediction Markets}
Prediction markets like Kalshi trade binary contracts that pay \$1 if an event occurs, otherwise \$0. Given the market price $P_K \in [0, 1]$ and assuming risk neutrality under the measure $\mathbb{Q}$, the price of a contract for an event $E$ at time $t$ with expiration $T$ is:
\begin{equation}
    P_K(t) = e^{-r(T-t)} \mathbb{E}^\mathbb{Q} [ \mathbb{I}_E \mid \mathcal{F}_t ]
\end{equation}
where $\mathbb{I}_E$ is the indicator function for the event. In practice, we treat Kalshi prices as approximately equal to probabilities for short-dated contracts, as the discount factor $e^{-r(T-t)}$ is negligible for near-term event maturities. Kalshi offers threshold contracts that imply a cumulative distribution. We recover the probability density function (PDF) by differencing adjacent contract thresholds, which serves as a discrete approximation of the derivative of the cumulative distribution function (CDF):
\begin{equation}
    f_K(x) \approx \frac{P_K(X > x) - P_K(X > x + \Delta x)}{\Delta x}
\end{equation}

\subsection{Risk-Neutral Density from Options Markets}
To derive a comparable signal from the traditional options market, we utilize the Black-Scholes framework. The probability that an option expires in-the-money (the risk-neutral probability) is given by the $N(d_2)$ component:
\begin{equation}
    P_{opt} = N(d_2) = N \left( \frac{\ln(S/K) + (r - \sigma^2/2)T}{\sigma\sqrt{T}} \right)
\end{equation}
where $\sigma$ is the implied volatility (IV) extracted from the volatility surface. In our empirical analysis, we use the VIX or VIX1D as a proxy for the short-term aggregate IV to measure institutional fear or expectation of a regime shift. Notably, as VIX is an aggregate volatility index rather than a strike-specific risk-neutral density, it serves as a coarse proxy for our comparisons.

\subsection{The Discrepancy Coefficient ($\lambda$)}
We define the \textit{Arb Gap} or Discrepancy Coefficient, $\lambda$, as the absolute difference between the Kalshi-implied probability and the options-implied probability:
\begin{equation}
    \lambda_t = | P_{Kalshi, t} - P_{Options, t} |
\end{equation}
A high $\lambda$ indicates a large gap between retail sentiment and institutional hedging behavior.

\subsection{Lead-Lag Analysis and Granger Causality}
To test whether Kalshi is a leading indicator of market volatility we use a Vector Autoregression (VAR) model and perform a Granger causality test. We assess whether or not past values of the Kalshi probability signal ($X$) help predict future values of the VIX ($Y$):
\begin{equation}
    Y_t = \alpha + \sum_{i=1}^{p} \beta_i Y_{t-i} + \sum_{j=1}^{p} \gamma_j X_{t-j} + \epsilon_t
\end{equation}
The null hypothesis $H_0: \gamma_1 = \gamma_2 = \dots = \gamma_p = 0$ is tested to evaluate whether the prediction market provides unique information prior to the options market "correcting" its volatility pricing.

\end{document}
